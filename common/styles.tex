\usepackage{alphalph}
\usepackage[inline]{enumitem}
\usepackage[super]{nth}
\usepackage{titlesec}
\usepackage{hyperref}

% Format article titles as "Article <number> - <name>", e.g., "Article 1 - Name"
\titleformat{\section}{\normalfont\Large\bfseries}{Article \thesection\enspace\textendash\enspace}{0em}{}

% Format reference links to look like links in web pages
\hypersetup{colorlinks=true, linkcolor=blue}

% Rename "section" to "article" in references.
\renewcommand*\sectionautorefname{article}

% Rename "subsection" to "§" in references. We have to do some extra work to
% remove a space so we get "§14.3" instead of "§ 14.3"
\def\Snospace~{\S{}}
\renewcommand*\subsectionautorefname{\Snospace}

% Format subsection titles to be inline with the subsection text
\titleformat{\subsection}[runin]{\normalfont\bfseries}{\thesubsection}{1em}{}

% Allow letter-enumerated lists to grow to (aa), (ab), ..., (az), (ba), (bb), ...
\makeatletter
\def\enumalphalphcnt#1{\expandafter\@enumalphalphcnt\csname c@#1\endcsname}
\def\@enumalphalphcnt#1{\alphalph{#1}}
\makeatother
\AddEnumerateCounter{\enumalphalphcnt}{\@enumalphalphcnt}{aa}

% Define inline and non-inline versions of letter-enumerated lists
\newlist{inlinealphalist}{enumerate*}{1}
\setlist[inlinealphalist]{label=(\alph*)}
\newlist{alphalist}{enumerate}{1}
\setlist[alphalist]{nosep, label=(\enumalphalphcnt*)}

% Define convenient shorthand commands for commonly used terms that should be written consistently
\newcommand{\fortythird}{\nth{43}}                              % \fortythird   = "43rd"
